\documentclass[a4paper,10pt]{article}
\usepackage[utf8]{inputenc}
\usepackage{cite}

%opening
\title{Summaries of Survey Papers}
\author{Alexandre Sawczuk da Silva}

\begin{document}

\maketitle

\section{A Survey on Bio-inspired Algorithms for Web Service Composition \cite{wang2012survey}}
This paper discusses research endeavours on Web service composition using the following techniques:
Ant Colony Algorithm, Genetic Algorithm, and Particle Swarm Optimisation.
However, before presenting information on each of these techniques it gives an overview of how the
problem of Web service composition is represented (i.e. the models of web service composition).
The paper notes that one of the limitations of this field is that there is no standard way to compare
the performance of two different approaches, so it is difficult to tell whether one has an advantage
over the other.

\subsection{Terminology}
All systems that make use of Web services use the principle of \textit{late binding}. According to this
principle, the functional needs to be fulfilled by a given service are represented by an abstract service,
and at runtime this is replaced by a concrete instance, often based on QoS measures. A composition can be
created either using \textit{local optimisation}, when the performance of each service in the composition
must be within a specified standard, or using \textit{global optimisation}, when the performance of the
overall composition must meet a given specification.  Creating a service composition with global constraints
is regarded as an NP-hard problem. Solutions to compositions can be calculated using \textit{exhaustive}
(i.e. deterministic) methods, which are expensive and scale poorly but yield the best
composition, or \textit{approximate} (i.e. nondeterministic) methods, which are cheaper but cannot
guarantee that the best composition will be found.

\subsection{Models for Web Service Composition}
At its core, Web Service Composition can be considered a \textit{combinatorial optimisation} problem, where
the objective is to find the best subset from a finite set of elements. Apart from \textit{graphs}, other representations 
for composition problems exist. These include the \textit{Knapsack problem}, where a number of items must be
selected to be a part of a solution set in order to maximise the total positive features of the set while keeping
the negative features at a minimum, and \textit{multi-objective/multi-dimensional} optimisation, where several
different goodness functions are considered simultaneously, leading to a Pareto set of results that presents different
trade-offs for each goodness measure. In this set, no result can be considered better than any other result, and no goodness
measure can be improved in a result without reducing some of its other goodness measures.

\subsection{Ant Colony Algorithm}
This algorithm is a representation of the behaviour of ants. These ants search optimal paths between a start and
an end node in a directed acyclic graph (DAG), and leave pheromones denoting the best composition path within the
DAG (note that the actual implementation of the pheromone mechanism varies from work to work, including the use of
different pheromones for different QoS attributes). In this approach, each atomic Web service is typically represented
as a node in the graph, and the QoS constraints are typically represented as weights applied to the graph edges.
However, different graph representations have also been presenting. Persisting on local optima is a potential problem
with this technique, and its efficiency is not the highest.

\subsection{Genetic Algorithm}
Genetic algorithms are a popular choice for tackling combinatorial optimisation problems, and thus have been widely
applied to the problem of Web service composition. A common representation for atomic Web services uses integer programming,
meaning that each service is represented as an integer. More specifially, linear integer programming is typically used,
restricting all constraints and fitness functions to be linear. The encoding scheme for a composition is commonly done 
as an array of integers, but some authors have attempted to use a matrix to also include semantic information (?). Researchers
also commonly investigate QoS representations, operators and fitness function variations to be applied to GA. An observed
problem with the GA technique is that it tends to prematurely converge to solutions, thus preventing it from exploring further
possibilities.

\subsection{Particle Swarm Optimisation}
Hybrid approaches have been attempted to improve the efficiency and optimisation power of PSO alone. Additionally, some 
authors have investigated the use of a PSO that is environment-aware. PSO may present the problem of not fully optimising
solutions (premature convergence).

\bibliography{summaries_surveys_bib}
\bibliographystyle{plain}
\end{document}
