\documentclass[a4paper,10pt]{article}
\usepackage[utf8]{inputenc}
\usepackage{cite}

%opening
\title{Summaries of Survey Papers}
\author{Alexandre Sawczuk da Silva}

\begin{document}

\maketitle

\section{A Survey on Bio-inspired Algorithms for Web Service Composition \cite{wang2012survey}}
This paper discusses research endeavours on Web service composition using the following techniques:
Ant Colony Algorithm, Genetic Algorithm, and Particle Swarm Optimisation.
However, before presenting information on each of these techniques it gives an overview of how the
problem of Web service composition is represented (i.e. the models of web service composition).
The paper notes that one of the limitations of this field is that there is no standard way to compare
the performance of two different approaches, so it is difficult to tell whether one has an advantage
over the other.

\subsection{Terminology}
All systems that make use of Web services use the principle of \textit{late binding}. According to this
principle, the functional needs to be fulfilled by a given service are represented by an abstract service,
and at runtime this is replaced by a concrete instance, often based on QoS measures. A composition can be
created either using \textit{local optimisation}, when the performance of each service in the composition
must be within a specified standard, or using \textit{global optimisation}, when the performance of the
overall composition must meet a given specification.  Creating a service composition with global constraints
is regarded as an NP-hard problem. Solutions to compositions can be calculated using \textit{exhaustive}
(i.e. deterministic) methods, which are expensive and scale poorly but yield the best
composition, or \textit{approximate} (i.e. nondeterministic) methods, which are cheaper but cannot
guarantee that the best composition will be found.

\subsection{Models for Web Service Composition}
At its core, Web Service Composition can be considered a \textit{combinatorial optimisation} problem, where
the objective is to find the best subset from a finite set of elements. Apart from \textit{graphs}, other representations 
for composition problems exist. These include the \textit{Knapsack problem}, where a number of items must be
selected to be a part of a solution set in order to maximise the total positive features of the set while keeping
the negative features at a minimum, and \textit{multi-objective/multi-dimensional} optimisation, where several
different goodness functions are considered simultaneously, leading to a Pareto set of results that presents different
trade-offs for each goodness measure. In this set, no result can be considered better than any other result, and no goodness
measure can be improved in a result without reducing some of its other goodness measures.

\subsection{Ant Colony Algorithm}
This algorithm is a representation of the behaviour of ants. These ants search optimal paths between a start and
an end node in a directed acyclic graph (DAG), and leave pheromones denoting the best composition path within the
DAG (note that the actual implementation of the pheromone mechanism varies from work to work, including the use of
different pheromones for different QoS attributes). In this approach, each atomic Web service is typically represented
as a node in the graph, and the QoS constraints are typically represented as weights applied to the graph edges.
However, different graph representations have also been presenting. Persisting on local optima is a potential problem
with this technique, and its efficiency is not the highest.

\subsection{Genetic Algorithm}
Genetic algorithms are a popular choice for tackling combinatorial optimisation problems, and thus have been widely
applied to the problem of Web service composition. A common representation for atomic Web services uses integer programming,
meaning that each service is represented as an integer. More specifially, linear integer programming is typically used,
restricting all constraints and fitness functions to be linear. The encoding scheme for a composition is commonly done 
as an array of integers, but some authors have attempted to use a matrix to also include semantic information (?). Researchers
also commonly investigate QoS representations, operators and fitness function variations to be applied to GA. An observed
problem with the GA technique is that it tends to prematurely converge to solutions, thus preventing it from exploring further
possibilities.

\subsection{Particle Swarm Optimisation}
Hybrid approaches have been attempted to improve the efficiency and optimisation power of PSO alone. Additionally, some 
authors have investigated the use of a PSO that is environment-aware. PSO may present the problem of not fully optimising
solutions (premature convergence).

\section{Service Selection in Web Service Composition: A Comparative Review of Existing Approaches \cite{moghaddam2014service}}
There exist two main approaches for automated Web service composition: \textit{workflow-based approaches}
and \textit{AI planning-based approaches}. In workflow-based approaches, the idea of an abstract business process that is to be
completed with concrete services is central. AI planning-based approaches, on the other hand, do not typically require this
abstract view. Composite service selection is considered an /textit{NP-hard problem}, meaning that the solution to large problems
is not likely to be found with reasonable computation times. This is typically addressed by reducing the search space.


\subsection{Lifecycle of a Typical Workflow-based WSC Solution}
A workflow-based WSC solution involves the following steps:

\begin{enumerate}
 \item \textit{Goal specification:} The user's goal and preferences for the composition are specified. An abstract workflow is generated,
 showing details about data flow and functionality, and the QoS requirements are determined based on the user's information.
 \item \textit{Service discovery:} Candidate concrete services that are functionally and non-functionally suitable to fill the slots
 in the abstract workflow are discovered in a service repository. At this stage, candidates have varying quality.
 \item \textit{Service selection:} A technique is employed to find which discovered services best fulfil each slot in the abstract
 workflow specified earlier, and a concrete Web service composition is generated.
 \item \textit{Service exection:} An instance of the concrete composition generated above is made executed.
 \item \textit{Service maintenance and monitoring:} The created instance is constantly monitored for failures and/or changes
 to the composing atomic services.
\end{enumerate}

\subsection{Optimisation-based WSC Approaches}
These approaches assume that QoS settings have already been made and are unchangeable during the period when service selection
takes places. They can also be referred to as \textit{QoS-aware} WSC approaches. One possible approach is \textit{Integer Linear
Programming}, where integers denote which services have been chosen to execute a specific task. This approach can be executed using
existing solvers, but this representation does not scale well. \textit{Genetic algorithms} are another possibility. In this case,
solutions can either be represented as a single dimension (a single execution path in the composition) or as a matrix (all execution
paths in the composition). GA is unconstrained, which may lead to solutions that are not functionally valid, therefore penalties
must be included into the fitness function employed. \textit{Constraint optimization} is another way to conceptualise WSC.
The algorithm in this technique is run according to a collection of constraints, searching for the solution that fits within
those constraints. This method is thought to be more scalable than those previously discussed. Finally, \textit{stochastic
programming} is discussed. This approach takes the non-deterministic nature of QoS attributes into account (e.g. variations in
response time, price, etc) by using a risk measure. The objective function then incorporates this measure in the creation of
a risk-aware composition. The disadvantage of this type of approach is that it does not take the dynamic nature of Web service
environments into account, assuming constancy in QoS values.

\subsection{Negotiation-based WSC Approaches}
These approaches assume that QoS settings can be modified after they have been initially set. In the context of Web service composition,
automated negotiation is used both for adjusting the details of Service Level Agreement contracts and for dynamically selecting the best
provider for particular Web services. This type of negotiation requires an organised framework, including a negotiation object (the service
with QoS attributes), a standardised protocol, and a decision-making model that implements a negotiation strategy. While this type of
approach reflects the dynamic environment of Web services, it requires a complex framework in order to be executed.

\subsection{Hybrid Approaches}
These approaches are not based purely on optimisation or negotiation, and two of them are discussed in this paper. The first one combines
\textit{optimisation and configuration}, that is, it executes an optimisation but allows users to specify ranges of values that are accepted for
each QoS attribute (as opposed to a single value). The second one combines \textit{optimisation and negotiation}, that is, if the optimisation
fails to encounter a solution with the requested QoS standards, a negotiation process is started.

\bibliography{summaries_surveys_bib}
\bibliographystyle{plain}
\end{document}
